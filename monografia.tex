\documentclass[12pt]{article}
\usepackage[a4paper,margin=1in,footskip=0.25in]{geometry} % set margins
\usepackage[portuguese]{babel}
\usepackage[utf8]{inputenc}
\usepackage{hyperref} 
\usepackage{amsmath}
\usepackage{amssymb}
\usepackage{amsthm}
\usepackage{graphicx}    % needed for include graphics
\usepackage{indentfirst}
\usepackage{float}       % needed for [H] figure placement option
\usepackage{setspace}    % needed for doublespacing

% Macros
\renewcommand{\familydefault}{\sfdefault} % sans-serif


\title{Arquitetura do Console Nintendo 64}
\author{Antônio Augusto Abello, Gustavo Estrela de Matos, Lucas Romão
Silva}

\begin{document}
% Espaçamento duplo 
\doublespacing
\begin{titlepage}
    \vfill
    \begin{center}
        \vspace{0.5\textheight}
        \noindent
        Instituto de Matemática e Estatística \\
        Monografia dos curso Organização de Computadores \\
        \vfill
        \noindent
        {\Large Arquitetura do Console Nintendo 64} \\
        \begin{tabular}{rl}
            {\bf Professor:} & {Siang Wun Song} \\
            {\bf Alunos:}    & {Antônio Augusto Abello} \\
                             & {Gustavo Estrela de Matos} \\
                             & {Lucas Romão Silva} \\
        \end{tabular} \\
        \vspace{\fill}
       \bigskip
        São Paulo, \today \\
       \bigskip
    \end{center}
\end{titlepage}

\pagebreak
\tableofcontents
\pagebreak

\section{Introdução}
\section{Principais Componentes do Nintendo64}
\section{Chip NEC VR4300}
    O chip NEC VR4300 é o principal processador no Nintendo64,
responsável principalmente por processar a lógica dos jogos e, 
também audio. Esse processador foi desenvolvido pela empresa japonesa
NEC e implementa a arquitetura de conjunto de instruções MIPS, 
desenvolvida pela empresa de mesmo nome. A arquitetura MIPS define um
conjunto de instruções do tipo RISC, \emph{reduced instruction set
computer}.

    O processador VR4300 possuia uma arquitetura compatível com 
instruções de 64 bits, apesar de grande parte das instruções do
Nintendo 64 serem de apenas 32 bits. Especificamente nesse console, o
processador da NEC trabalhava a uma frenquência de 93.75 MHz.
\subsection{}
\section{Chip SGI RCP}

\end{document}


